\documentclass[polish,polish,a4paper]{article}
\usepackage[T1]{fontenc}
\usepackage[utf8]{inputenc}
\usepackage{pgfplots}
\usepackage{pslatex}
\usepackage{setspace}
\usepackage{caption}
\usepackage{amssymb}
\usepackage{amsmath}
\usepackage{anysize}
\usepackage{graphicx}
\usepackage{hyperref}
\usepackage{float}
\usepackage{color}
\usepackage{subcaption}
\usepackage[polish]{babel}
\hypersetup{
	colorlinks=true,
	linkcolor=blue,
	filecolor=blue,      
	urlcolor=blue,
}

\marginsize{2.5cm}{2.5cm}{2cm}{2cm}


\begin{document}
	
		\begin{titlepage}
			\begin{flushright}
				{ Wtorki 16:50\\
					Grupa I3\\
					Kierunek Informatyka\\
					Wydział Informatyki\\
					Politechnika Poznańska}
			\end{flushright}
		\vspace*{\fill}
		\begin{center}
			{\Large Algorytmy i struktury danych \\[0.1cm]
				Sprawozdanie z zadania w zespołach nr. 4\\[0.1cm]
				prowadząca: dr hab. inż. Małgorzata Sterna, prof PP}\\
			{\Huge Programowanie dynamiczne\\ [0.4cm]}
			{\large autorzy:\\[0.1cm]}
			{\large Piotr Więtczak nr indeksu 132339\\[0.1cm] Tomasz Chudziak nr indeksu 136691}\\[0.5cm]
			\today
		\end{center}
		\vspace*{\fill}
	\end{titlepage}

\begin{spacing}{1.25}
	
	\section{Implementacja}
	\section{Zależność czasu obliczeń t od liczby paczek n dla $PD$, $BF_{1}$, $BF_{2}$, $GH_{4}$ dla $b = 50 \% \sum s (a_{i})$. }
	\section{Zależność czasu obliczeń t od liczby paczek n dla metody $PD$, $BF_{1}$, $BF_{2}$, $GH_{4}$ dla $b = 25 \% \sum s (a_{i})$ i $b = 75 \% \sum s (a_{i})$. }
	\section{Obliczanie ś☺redniego błędu popełnionego przez h}


\end{spacing}
	\newpage
	\tableofcontents
\end{document}


