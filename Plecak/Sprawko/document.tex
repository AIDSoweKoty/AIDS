\documentclass[polish,polish,a4paper]{article}
\usepackage[T1]{fontenc}
\usepackage[utf8]{inputenc}
\usepackage{pgfplots}
\usepackage{pslatex}
\usepackage{setspace}
\usepackage{caption}
\usepackage{amssymb}
\usepackage{amsmath}
\usepackage{anysize}
\usepackage{graphicx}
\usepackage{hyperref}
\usepackage{float}
\usepackage{color}
\usepackage{subcaption}
\usepackage[polish]{babel}
\hypersetup{
	colorlinks=true,
	linkcolor=blue,
	filecolor=blue,      
	urlcolor=blue,
}

\marginsize{2.5cm}{2.5cm}{2cm}{2cm}


\begin{document}
	
		\begin{titlepage}
			\begin{flushright}
				{ Wtorki 16:50\\
					Grupa I3\\
					Kierunek Informatyka\\
					Wydział Informatyki\\
					Politechnika Poznańska}
			\end{flushright}
		\vspace*{\fill}
		\begin{center}
			{\Large Algorytmy i struktury danych \\[0.1cm]
				Sprawozdanie z zadania w zespołach nr. 4\\[0.1cm]
				prowadząca: dr hab. inż. Małgorzata Sterna, prof PP}\\
			{\Huge Programowanie dynamiczne\\ [0.4cm]}
			{\large autorzy:\\[0.1cm]}
			{\large Piotr Więtczak nr indeksu 132339\\[0.1cm] Tomasz Chudziak nr indeksu 136691}\\[0.5cm]
			\today
		\end{center}
		\vspace*{\fill}
	\end{titlepage}

\begin{spacing}{1.25}
	
	\section{Heurystyki}
	
	Algorytm heurystyczny jest to schemat postępowania, który nie gwarantuje odnalezienia najlepszego rozwiązania, lecz dopuszczalnie dobrego. Oznacza to, że krótszy czas poświęcony na uzyskanie go, pozwala zrekompensować korzyści płynące z najlepszego. Algorytm zachłanny charakteryzuje się tym, że w każdym kroku decyzje podejmuje zachłannie. Innymi słowy, stara się on podjąć jak najlepszą decyzję na daną chwilę, nie dokonując oceny czy będzie to korzystne w kolejnych krokach. Algorytmy listowe wpierw porządkują dane wg jakiegoś kryterium (priorytet, wartość). Później rozpatruje je wg ustalonej kolejności.
	
	W naszym badaniu testowaliśmy cztery heurytyki różniące się regułami wyboru paczek: losową ($GH_{1}$), min\{s($a_{i}$)\} ($GH_{2}$), max\{w($a_{i}$)\} ($GH_{3}$) i max\{w($a_{i}$)/s($a_{i}$)\} ($GH_{4}$). Sprawdzaliśmy błąd popełniany przez poszczególne heurystyki dla różnych ładowności pojazdu. 
	
	
	\subsection{ Błąd popełniany przez poszczególne heurystyki dla różnych wartości n gdy $b = 25 \% \sum s (a_{i})$
		 :}
	\begin{spacing}{1.5}
		\begin{equation*}
		\begin{array}{|c|c|c|c|c|}
		\hline
		Liczba paczek&$$GH_{1}$$&$$GH_{2}$$&$$GH_{3}$$&$$GH_{4}$$\\\hline
		10&29.5175&17.2924&10.7574&17.7376\\ \hline
		12&55.0858&14.7939&23.6520&5.2983\\\hline
		14&33.7494&22.4527&30.5146&0.6605\\\hline
		16&65.0744&19.3239&10.2672&0.5685\\\hline
		18&56.9515&23.3476&30.5126&0\\\hline
		20&56.4759&15.8717&33.0602&3.6663\\\hline
		22&30.2727&20.9886&18.1095&4.8062\\\hline
		24&46.826&20.2912&20.2350&0\\\hline
		26&52.6595&19.7696&18.3623&1.4509\\\hline
		28&54.9991&22.3583&23.7379&0\\\hline
		30&40.4368&22.8105&17.9674&1.2004\\\hline
		średnia&52.2048&21.93&23.7176&3.5389\\
		\hline
		\end{array}
		\end{equation*}
	\end{spacing}

\subsection{ Błąd popełniany przez poszczególne heurystyki dla różnych wartości n gdy $b = 50 \% \sum s (a_{i})$
	:}
\begin{spacing}{1.5}
	\begin{equation*}
	\begin{array}{|c|c|c|c|c|}
	\hline
	Liczba paczek&$$GH_{1}$$&$$GH_{2}$$&$$GH_{3}$$&$$GH_{4}$$\\ \hline
	10&67.7162&16.4797&17.9785&0\\ \hline
	12&42.9433&24.6597&16.2328&0.9283\\ \hline
	14&18.2295&18.1809&12.7344&0.2793\\ \hline
	16&53.4545&20.2261&10.3773&9.0687\\ \hline
	18&51.3527&13.2657&5.2776&4.2163\\ \hline
	20&41.0811&13.3351&5.3131&0.8656\\ \hline
	22&36.206&14.8905&3.9989&3.6787\\ \hline
	24&29.008&15.5245&6.0858&5.8083\\ \hline
	26&49.571&12.3309&7.9051&1.4672\\ \hline
	28&24.4841&13.1157&2.2750&0\\ \hline
	30&41.7705&17.1165&11.7752&2.8098\\ \hline
	średnia&45.5817&17.9125&9.99541&2.91228\\ \hline
	\end{array}
	\end{equation*}
\end{spacing}

\subsection{ Błąd popełniany przez poszczególne heurystyki dla różnych wartości n gdy $b = 75 \% \sum s (a_{i})$
	:}
\begin{spacing}{1.5}
	\begin{equation*}
	\begin{array}{|c|c|c|c|c|}
	\hline
	Liczba_paczek&$$GH_{1}$$&$$GH_{2}$$&$$GH_{3}$$&$$GH_{4}$$\\\hline
	10&47.2733&0.8288&1.4072&2.2360\\\hline
	12&55.0922&0.6372&1.0819&1.7192\\\hline
	14&16.0029&0.5830&2.4653&3.0484\\\hline
	16&22.0284&10.4946&0&0\\\hline
	18&9.7660&6.3365&1.8275&2.2598\\\hline
	20&16.0535&7.6278&0.9253&1.2910\\\hline
	22&28.2559&6.4437&0.5193&0.5631\\\hline
	24&12.068&8.4404&2.4033&3.0518\\\hline
	26&17.4&18.1838&7.9579&0.4409\\\hline
	28&16.7216&5.5104&0.1375&0.9379\\\hline
	30&26.3282&10.1727&1.1162&0.4336\\\hline
	średnia&26.699&7.5259&1.9841&1.5982\\\hline
	\end{array}
	\end{equation*}
\end{spacing}
\subsection{Podsumowanie}

	\begin{spacing}{1.5}
		\begin{equation*}
		\begin{array}{|c|c|c|c|c|}
			\hline
			śr. błąd&$$GH_{1}$$&$$GH_{2}$$&$$GH_{3}$$&$$GH_{4}$$\\ \hline
			$b=25\%$&52.2048&21.93&23.7176&3.5389\\\hline
			$b=50\%$&45.5817&17.9125&9.9951&2.9122\\\hline
			$b=75\%$&26.699&7.5259&1.9841&1.5982\\\hline
			średnia&46.8020&15.7895&11.8990&2.7031\\
			\hline
		\end{array}
		\end{equation*}
	\end{spacing}
	
	Wraz ze wzrostem ładowności pojazdu algorytmy radzą sobie coraz lepiej. Jest to spowodowane tym, że mniejsza ilość paczek nie zostanie użyta. Dla heurystyk ich wydajność niewydane się być skorelowana z ilością paczek, z którą ma do czynienia.
	
	Jak widać w tabeli powyżej, jakość rozwiązanie zależy od użytej metody. Spośród badanych metod najmniej efektywna okazała się $GH_{1}$, następnie $GH_{2}$. Najlepiej wypadła $GH_{4}$ a po niej $GH_{3}$. Wydajność $GH_{1}$ zależy tylko i wyłącznie od przypadku. Skuteczność $GH_{2}$ i $GH_{3}$ jest zbliżona, pierwsza sortuje wg rozmiaru rosnąco, natomiast drugi wg rozmiaru malejąco. $GH_{2}$ będzie radził sobie gorzej dla dużej ilości małych paczek o niskiej wartości, a $GH_{3}$ dla stosunkowo dużej ilości dużych paczek o wysokiej wartości. $GH_{4}$ bierze pod uwagę stosunek wartości do rozmiaru paczki, co w konsekwencji daje  wynik najbardziej zbliżony do optymalnego. 
	
	


































	
\section{Zależność czasu obliczeń t od liczby paczek n dla $PD$, $BF_{1}$, $BF_{2}$, $GH_{4}$ dla $b = 25 \% \sum s (a_{i})$. }
	
	\begin{spacing}{1.5}
		\begin{equation*}
		\begin{array}{|r|r|r|r|r|}
		\hline
		\multicolumn{1}{|c|}{$Liczba paczek$}&	\multicolumn{4}{c|}{Metody}\\\cline{2-5}
		\multicolumn{1}{|c|}{$paczek$}
		&\multicolumn{1}{c|}{$PD$}
		&\multicolumn{1}{c|}{$$BF_{1}$$}
		&\multicolumn{1}{c|}{$$BF_{2}$$}
		&\multicolumn{1}{c|}{$GH$}
		\\\hline
		10& 2.02074& 0.814474& 1.23603& 0.349975\\
		12& 1.39877& 0.921087& 1.4534& 0.42984\\
		14& 1.57056& 1.38031& 1.51066& 0.433231\\
		16& 1.777& 2.7998& 1.81769& 0.485972\\
		18& 2.19479& 9.5891& 2.3952& 0.55642\\
		20& 2.29349& 35.3227& 4.15186 &0.528542\\
		22& 2.44305 &118.491& 6.80211 &0.489739\\
		24& 2.25205& 540.134 &20.7631 &0.497274\\
		26& 3.04505& 2175.09 &71.9876& 0.507069\\
		28& 3.45681& 7193.46 &114.97 &0.435491\\
		30& 3.61663& 27904.3 &318.769& 0.411758\\
		\hline
		\end{array}
		\end{equation*}
	\end{spacing}
	
	\begin{figure}[H]
		\centering
		\begin{tikzpicture}
		\begin{axis}[
		width=0.9\textwidth,
		height = 0.5\textwidth,
		title={Czasy znajdowania najlepszego ułożenia w zależności od ilości paczek.},
		xlabel={Liczba paczek},
		ylabel={Czas zliczania w misekundach},
		%
		scaled x ticks = false,
		xtick distance = 2,
		x tick label style={/pgf/number format/fixed},
		xticklabel style = {rotate= 90},
		x label style={at={(axis description cs:0.5,-0.15)},anchor=north},
		%%
		ymode = log,
		ytick distance = 10,
		scaled y ticks = false,
		y tick label style={/pgf/number format/fixed},
		y label style={at={(axis description cs:-0.05,0.85)},anchor=east},
		%%
		legend pos=north west,
		ymajorgrids=true,
		grid style=dashed,
		]
		%%
		\addplot[
		color=black,
		mark=*,
		]
		coordinates {
			(10,2.02074 )
			(12,1.39877 )
			(14,1.57056 )
			(16,1.777 )
			(18,2.19479 )
			(20,2.29349 )
			(22,2.44305 )
			(24,2.25205 )
			(26,3.04505 )
			(28,3.45681 )
			(30,3.61663 )
		};
		%%
		\addplot[
		color=red,
		mark=o,
		]
		coordinates {
			(10,0.814474 )
			(12,0.921087 )
			(14,1.38031 )
			(16,2.7998 )
			(18,9.5891 )
			(20,35.3227 )
			(22,118.491 )
			(24,540.134 )
			(26,2175.09 )
			(28,7193.46 )
			(30,27904.3 )
		};
		
		%%
		\addplot[
		color=blue,
		mark=x,
		]
		coordinates {
			(10,1.23603 )
			(12,1.4534 )
			(14,1.51066 )
			(16,1.81769 )
			(18,2.3952 )
			(20,4.15186 )
			(22,6.80211 )
			(24,20.7631 )
			(26,71.9876 )
			(28,114.97 )
			(30,318.769 )
		};
					%%
		\addplot[
		color=green,
		mark=otimes,
		]
		coordinates {
			(10,0.349975)
			(12,0.42984)
			(14,0.433231)
			(16,0.485972)
			(18,0.55642)
			(20,0.528542)
			(22,0.489739)
			(24,0.497274)
			(26,0.507069)
			(28,0.435491)
			(30,0.411758)
		};
		\legend{
			PD,
			$BF_{1}$,
			$BF_{2}$,
			GH	
		}
		%%
		\end{axis}
		\end{tikzpicture}
	\end{figure}
	
	
\section{Zależność czasu obliczeń t od liczby paczek n dla $PD$, $BF_{1}$, $BF_{2}$, $GH_{4}$ dla $b = 50 \% \sum s (a_{i})$. }
	
		\begin{spacing}{1.5}
		\begin{equation*}
		\begin{array}{|r|r|r|r|r|}
		\hline
		\multicolumn{1}{|c|}{$Liczba paczek$}&	\multicolumn{4}{c|}{Metody}\\\cline{2-5}
		\multicolumn{1}{|c|}{$paczek$}
		&\multicolumn{1}{c|}{$PD$}
		&\multicolumn{1}{c|}{$$BF_{1}$$}
		&\multicolumn{1}{c|}{$$BF_{2}$$}
		&\multicolumn{1}{c|}{$GH$}
		\\\hline
		10&1.56491&0.903381&1.70618&0.450937\\
		12&1.7269&0.899991&1.33586&0.380867\\
		14&1.80375&1.14335&3.41386&0.675087\\
		16&2.21588&2.43325&2.816&0.404977\\
		18&3.15129&8.73093&5.96503&0.388401\\
		20&2.72559&28.9478&20.1306&0.391415\\
		22&3.1317&110.618&66.8581&0.423813\\
		24&3.7224&441.764&285.001&0.39895\\
		26&3.90034&1757.5&1081.15&0.427204\\
		28&4.62126&6933.11&4051.2&0.400457\\
		30&5.54084&27831.4&16303.1&0.526658\\
		\hline
		\end{array}
		\end{equation*}
	\end{spacing}
	
	\begin{figure}[H]
		\centering
		\begin{tikzpicture}
		\begin{axis}[
		width=0.9\textwidth,
		height = 0.5\textwidth,
		title={Czasy znajdowania najlepszego ułożenia w zależności od ilości paczek.},
		xlabel={Liczba paczek},
		ylabel={Czas zliczania w misekundach},
		%
		scaled x ticks = false,
		xtick distance = 2,
		x tick label style={/pgf/number format/fixed},
		xticklabel style = {rotate= 90},
		x label style={at={(axis description cs:0.5,-0.15)},anchor=north},
		%%
		ymode = log,
		ytick distance = 10,
		scaled y ticks = false,
		y tick label style={/pgf/number format/fixed},
		y label style={at={(axis description cs:-0.05,0.85)},anchor=east},
		%%
		legend pos=north west,
		ymajorgrids=true,
		grid style=dashed,
		]
		%%
		\addplot[
		color=black,
		mark=*,
		]
		coordinates {
			(10,1.56491 )
			(12,1.7269 )
			(14,1.80375 )
			(16,2.21588 )
			(18,3.15129 )
			(20,2.72559 )
			(22,3.1317 )
			(24,3.7224 )
			(26,3.90034 )
			(28,4.62126 )
			(30,5.54084 )
		};
		%%
		\addplot[
		color=red,
		mark=o,
		]
		coordinates {
			(10,0.903381 )
			(12,0.899991 )
			(14,1.14335 )
			(16,2.43325 )
			(18,8.73093 )
			(20,28.9478 )
			(22,110.618 )
			(24,441.764 )
			(26,1757.5 )
			(28,6933.11 )
			(30,27831.4 )
		};
		
		%%
		\addplot[
		color=blue,
		mark=x,
		]
		coordinates {
			
			(10,1.70618 )
			(12,1.33586 )
			(14,3.41386 )
			(16,2.816 )
			(18,5.96503 )
			(20,20.1306 )
			(22,66.8581 )
			(24,285.001 )
			(26,1081.15 )
			(28,4051.2 )
			(30,16303.1 )
		};
		%%
		\addplot[
		color=green,
		mark=otimes,
		]
		coordinates {
			(10,0.450937)
			(12,0.380867)
			(14,0.675087)
			(16,0.404977)
			(18,0.388401)
			(20,0.391415)
			(22,0.423813)
			(24,0.39895)
			(26,0.427204)
			(28,0.400457)
			(30,0.526658)
		};
		\legend{
			PD,
			$BF_{1}$,
			$BF_{2}$,
			GH	
		}
		%%
		\end{axis}
		\end{tikzpicture}
	\end{figure}
	

\section{Zależność czasu obliczeń t od liczby paczek n dla metody $PD$, $BF_{1}$, $BF_{2}$, $GH_{4}$ dla $b = 25 \% \sum s (a_{i})$ i $b = 75 \% \sum s (a_{i})$. }
	
		\begin{spacing}{1.5}
		\begin{equation*}
		\begin{array}{|r|r|r|r|r|}
		\hline
		\multicolumn{1}{|c|}{$Liczba paczek$}&	\multicolumn{4}{c|}{Metody}\\\cline{2-5}
		\multicolumn{1}{|c|}{$paczek$}
		&\multicolumn{1}{c|}{$PD$}
		&\multicolumn{1}{c|}{$$BF_{1}$$}
		&\multicolumn{1}{c|}{$$BF_{2}$$}
		&\multicolumn{1}{c|}{$GH$}
		\\\hline
		10&2.0181&0.714266&1.85724&0.430594\\
		12&2.42572&0.832934&1.60484&0.407237\\
		14&2.5474&1.1539&2.2528&0.478061\\
		16&3.14187&2.40048&3.32157&0.434738\\
		18&7.15246&8.19259&9.35064&0.480321\\
		20&4.34399&29.3411&30.5944&0.460355\\
		22&4.94298&113.809&120.64&0.481075\\
		24&5.71451&434.529&468.442&0.467889\\
		26&8.79798&1905.24&1905.06&0.417785\\
		28&5.9379&7045.24&7187.44&0.479568\\
		30&6.1014&27492.2&28313.4&0.404224\\
		\hline
		\end{array}
		\end{equation*}
	\end{spacing}
	
	\begin{figure}[H]
		\centering
		\begin{tikzpicture}
		\begin{axis}[
		width=0.9\textwidth,
		height = 0.5\textwidth,
		title={Czasy znajdowania najlepszego ułożenia w zależności od ilości paczek.},
		xlabel={Liczba paczek},
		ylabel={Czas zliczania w misekundach},
		%
		scaled x ticks = false,
		xtick distance = 2,
		x tick label style={/pgf/number format/fixed},
		xticklabel style = {rotate= 90},
		x label style={at={(axis description cs:0.5,-0.15)},anchor=north},
		%%
		ymode = log,
		ytick distance = 10,
		scaled y ticks = false,
		y tick label style={/pgf/number format/fixed},
		y label style={at={(axis description cs:-0.05,0.85)},anchor=east},
		%%
		legend pos=north west,
		ymajorgrids=true,
		grid style=dashed,
		]
		%%
		\addplot[
		color=black,
		mark=*,
		]
		coordinates {
			(10,2.0181 )
			(12,2.42572 )
			(14,2.5474 )
			(16,3.14187 )
			(18,7.15246 )
			(20,4.34399 )
			(22,4.94298 )
			(24,5.71451 )
			(26,8.79798 )
			(28,5.9379 )
			(30,6.1014 )
		};
		%%
		\addplot[
		color=red,
		mark=o,
		]
		coordinates {
			(10,0.714266 )
			(12,0.832934 )
			(14,1.1539 )
			(16,2.40048 )
			(18,8.19259 )
			(20,29.3411 )
			(22,113.809 )
			(24,434.529 )
			(26,1905.24 )
			(28,7045.24 )
			(30,27492.2 )
		};
		
		%%
		\addplot[
		color=blue,
		mark=x,
		]
		coordinates {
			(10,1.85724 )
			(12,1.60484 )
			(14,2.2528 )
			(16,3.32157 )
			(18,9.35064 )
			(20,30.5944 )
			(22,120.64 )
			(24,468.442 )
			(26,1905.06 )
			(28,7187.44 )
			(30,28313.4 )
		};
		%%
		\addplot[
		color=green,
		mark=otimes,
		]
		coordinates {
			(10,0.430594)
			(12,0.407237)
			(14,0.478061)
			(16,0.434738)
			(18,0.480321)
			(20,0.460355)
			(22,0.481075)
			(24,0.467889)
			(26,0.417785)
			(28,0.479568)
			(30,0.404224)
		};
		\legend{
			PD,
			$BF_{1}$,
			$BF_{2}$,
			GH	
		}
		%%
		\end{axis}
		\end{tikzpicture}
	\end{figure}
	
	\section{Obliczanie średniego błędu popełnionego przez h}


\end{spacing}
	\newpage
	\tableofcontents
\end{document}


