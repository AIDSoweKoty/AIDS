\documentclass[polish,polish,a4paper]{article}
\usepackage[T1]{fontenc}
\usepackage[utf8]{inputenc}
\usepackage{pgfplots}
\usepackage{pslatex}
\usepackage{setspace}
\usepackage{caption}
\usepackage{amssymb}
\usepackage{amsmath}
\usepackage{anysize}
\usepackage{graphicx}
\usepackage{hyperref}
\usepackage{float}
\usepackage[polish]{babel}
\hypersetup{
	colorlinks=true,
	linkcolor=blue,
	filecolor=blue,      
	urlcolor=blue,
}

\marginsize{2.5cm}{2.5cm}{2cm}{2cm}


\begin{document}
	
		\begin{titlepage}
			\begin{flushright}
				{ Wtorki 16:50\\
					Grupa I3\\
					Kierunek Informatyka\\
					Wydział Informatyki\\
					Politechnika Poznańska}
			\end{flushright}
		\vspace*{\fill}
		\begin{center}
			{\Large Algorytmy i struktury danych \\[0.1cm]
				Sprawozdanie z zadania w zespołach nr. 2\\[0.1cm]
				prowadząca: dr hab. inż. Małgorzata Sterna, prof PP}\\
			{\Huge Algorytmy  Grafowe\\ [0.4cm]}
			{\large autorzy:\\[0.1cm]}
			{\large Piotr Więtczak nr indeksu 132339\\[0.1cm] Tomasz Chudziak nr indeksu 136691}\\[0.5cm]
			\today
		\end{center}
		\vspace*{\fill}
	\end{titlepage}

\section{Opis implementacji }

\section{Obliczanie etykiet}
	%%WYKRES OBLICZANIA ETYKIET
	\begin{figure}[H]
		\centering
		\begin{tikzpicture}
		\begin{axis}[
		width=0.9\textwidth,
		height = 0.5\textwidth,
		title={Czasy obliczania etykiet dla grafów},
		xlabel={Liczba wierzchołków},
		ylabel={Czas obliczania w misekundach},
		%
		scaled x ticks = false,
		xtick distance = 200,
		x tick label style={/pgf/number format/fixed},
		xticklabel style = {rotate= 90},
		x label style={at={(axis description cs:0.5,-0.15)},anchor=north},
		%%
		ytick distance = 100,
		scaled y ticks = false,
		y tick label style={/pgf/number format/fixed},
		y label style={at={(axis description cs:-0.05,0.85)},anchor=east},
		%%
		legend pos=north west,
		ymajorgrids=true,
		grid style=dashed,
		]
		%%
		\addplot[
		color=black,
		mark=*,
		]
		coordinates {
(200,2.172)(400,7.830)(600,17.860)(800,31.477)(1000,55.936)(1200,70.922)(1400,91.161)(1600,117.204)(1800,175.443)(2000,183.787)(2200,222.953)
		};
		%%
		\addplot[
		color=blue,
		mark=triangle,
		]
		coordinates {
(200,4.841)(400,15.622)(600,38.984)(800,72.013)(1000,92.017)(1200,135.757)(1400,179.157)(1600,265.713)(1800,295.831)(2000,367.042)(2200,616.094)

		};
		\legend{
			$d = 0.2$,
			$d = 0.4$,
		}
		%%
		\end{axis}
		\end{tikzpicture}
	\end{figure}
	
\section{Liczba łuków powrotnych}

\section{Czasy zliczania łuków powrotnych}
%WYKRES DLA 0.2
	\begin{figure}[H]
	\centering
	\begin{tikzpicture}
	\begin{axis}[
	width=0.9\textwidth,
	height = 0.5\textwidth,
	title={Czasy zliczania łuków powrotnych dla grafu o gestości $ d = 0.2 $},
	xlabel={Liczba wierzchołków},
	ylabel={Czas zliczania w misekundach},
	%
	scaled x ticks = false,
	xtick distance = 200,
	x tick label style={/pgf/number format/fixed},
	xticklabel style = {rotate= 90},
	x label style={at={(axis description cs:0.5,-0.15)},anchor=north},
	%%
	ytick distance = 10,
	scaled y ticks = false,
	y tick label style={/pgf/number format/fixed},
	y label style={at={(axis description cs:-0.05,0.85)},anchor=east},
	%%
	legend pos=north west,
	ymajorgrids=true,
	grid style=dashed,
	]
	%%
	\addplot[
	color=black,
	mark=*,
	]
	coordinates {
(200,0.25600)(400,0.86755500)(600,1.8375100)(800,3.2324200)(1000,4.9863100)(1200,7.6993400)(1400,9.8605400)(1600,12.549100)(1800,16.112600)(2000,19.719900)(2200,24.805800)

	};
	%%
	\addplot[
	color=blue,
	mark=triangle,
	]
	coordinates {
(200,0.37034600)(400,1.2800)(600,2.9246600)(800,5.1700600)(1000,7.3636900)(1200,11.28500)(1400,14.333700)(1600,19.03500)(1800,26.807200)(2000,30.5300)(2200,37.193900)

	};
	%%
\addplot[
color=red,
mark=o,
]
coordinates {
(200,0.34190200)(400,2.4513400)(600,2.934900)(800,4.9965500)(1000,7.1987100)(1200,11.526800)(1400,15.494800)(1600,19.957700)(1800,31.353700)(2000,32.027800)(2200,40.884300)

};
	\legend{
		macierz grafu,
		lista następników,
		lista łuków,
	}
	%%
	\end{axis}
	\end{tikzpicture}
\end{figure}

%WYKRES DLA 0.4
\begin{figure}[H]
	\centering
	\begin{tikzpicture}
	\begin{axis}[
	width=0.9\textwidth,
	height = 0.5\textwidth,
	title={Czasy zliczania łuków powrotnych dla grafu o gestości $ d = 0.4 $},
	xlabel={Liczba wierzchołków},
	ylabel={Czas zliczania w misekundach},
	%
	scaled x ticks = false,
	xtick distance = 200,
	x tick label style={/pgf/number format/fixed},
	xticklabel style = {rotate= 90},
	x label style={at={(axis description cs:0.5,-0.15)},anchor=north},
	%%
	ytick distance = 10,
	scaled y ticks = false,
	y tick label style={/pgf/number format/fixed},
	y label style={at={(axis description cs:-0.05,0.85)},anchor=east},
	%%
	legend pos=north west,
	ymajorgrids=true,
	grid style=dashed,
	]
	%%
	\addplot[
	color=black,
	mark=*,
	]
	coordinates {
(200,0.40391100)(400,1.3203900)(600,2.9451400)(800,5.2428700)(1000,8.1328300)(1200,12.375600)(1400,15.673400)(1600,20.920900)(1800,25.759300)(2000,32.119400)(2200,53.42200)

	};
	%%
	\addplot[
	color=blue,
	mark=triangle,
	]
	coordinates {
(200,0.87324400)(400,3.141400)(600,5.4977400)(800,9.3798300)(1000,14.965700)(1200,21.315700)(1400,29.834200)(1600,38.461400)(1800,49.943900)(2000,60.332900)(2200,96.386200)


	};
	%%
	\addplot[
	color=red,
	mark=o,
	]
	coordinates {
(200,0.75832800)(400,2.7278200)(600,5.8015200)(800,9.4731300)(1000,17.002400)(1200,21.404400)(1400,34.082100)(1600,43.678700)(1800,55.228300)(2000,73.463400)(2200,96.399300)


	};
	\legend{
		macierz grafu,
		lista następników,
		lista łuków,
	}
	%%
	\end{axis}
	\end{tikzpicture}
\end{figure}

	\newpage
	\tableofcontents
\end{document}


