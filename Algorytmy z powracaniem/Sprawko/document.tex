\documentclass[polish,polish,a4paper]{article}
\usepackage[T1]{fontenc}
\usepackage[utf8]{inputenc}
\usepackage{pgfplots}
\usepackage{pslatex}
\usepackage{setspace}
\usepackage{caption}
\usepackage{amssymb}
\usepackage{amsmath}
\usepackage{anysize}
\usepackage{graphicx}
\usepackage{hyperref}
\usepackage{float}
\usepackage{color}
\usepackage{subcaption}
\usepackage[polish]{babel}
\hypersetup{
	colorlinks=true,
	linkcolor=blue,
	filecolor=blue,      
	urlcolor=blue,
}

\marginsize{2.5cm}{2.5cm}{2cm}{2cm}


\begin{document}
	
		\begin{titlepage}
			\begin{flushright}
				{ Wtorki 16:50\\
					Grupa I3\\
					Kierunek Informatyka\\
					Wydział Informatyki\\
					Politechnika Poznańska}
			\end{flushright}
		\vspace*{\fill}
		\begin{center}
			{\Large Algorytmy i struktury danych \\[0.1cm]
				Sprawozdanie z zadania w zespołach nr. 4\\[0.1cm]
				prowadząca: dr hab. inż. Małgorzata Sterna, prof PP}\\
			{\Huge Algorytmy z powracaniem\\ [0.4cm]}
			{\large autorzy:\\[0.1cm]}
			{\large Piotr Więtczak nr indeksu 132339\\[0.1cm] Tomasz Chudziak nr indeksu 136691}\\[0.5cm]
			\today
		\end{center}
		\vspace*{\fill}
	\end{titlepage}

\begin{spacing}{1.25}

\section{Opis implementacji }
 
 Do implementacji algorytmów poszukujących cyklu Eulera $(E)$, pojedynczego cyklu Hamiltona $(H1)$ i wszystkich cykli Hamiltona użyliśmy języka C++. Do pomiarów czasu wykorzystaliśmy klasę $std :: chrono :: high\_resolution\_clock$ z biblioteki $chrono$. Do reprezentacji grafu zastosowaliśmy macierz sąsiedztwa, ze względu na {\Huge \textcolor{red}{TU MI SIE TŁUMACZ CZEMU MACIERZ SĄSIEDZTWA}}.  

\section{Czasy działania algorytmów}


	%% TABELA CZASÓW DZIAŁANIA
\begin{spacing}{1.3}
	
	{
		\centering
		\subsection*{Tabela przedstawiająca czasy działania algorytmów }
	}
	
	\begin{equation*}
	\begin{array}{|r|r|r|r|}
	\hline
	\multicolumn{1}{|c|}{$Liczba$}&
	\multicolumn{1}{c|}{t_{E}$ dla$ }&
	\multicolumn{1}{c|}{t_{H1} $ dla$}&
	\multicolumn{1}{c|}{t_{HA}$ dla$}\\
	\multicolumn{1}{|c|}{$wierzchołków$}&
	\multicolumn{1}{c|}{d = 0.6 \quad [ms]}&
	\multicolumn{1}{c|}{d = 0.6 \quad [ms]}&
	\multicolumn{1}{c|}{d = 0.6 \quad [ms]}\\
	\hline
5&0.001&0.002&0.003\\
6&0.001&0.004&0.017\\
7&0.001&0.046&0.051\\
8&0.002&0.114&0.098\\
9&0.002&0.012&0.897\\
10&0.002&0.015&6.125\\
11&0.003&0.289&28.681\\
12&0.003&0.065&171.070\\
13&0.003&0.010&1486.260\\
14&0.003&0.016&8225.920\\
20&0.006&0.036&-\\
25&0.010&0.391&-\\
30&0.014&0.051&-\\
35&0.019&0.044&-\\
40&0.025&0.077&-\\
45&0.031&0.130&-\\
50&0.039&0.158&-\\
55&0.046&1.079&-\\
60&0.054&0.190&-\\
65&0.065&0.231&-\\
70&0.074&0.386&-\\
75&0.088&1.549&-\\
80&0.095&0.788&-\\
85&0.107&0.926&-\\
90&0.120&0.550&-\\
95&0.135&0.663&-\\
100&0.199&3.552&-\\\hline
	\end{array}
	\end{equation*}
\end{spacing}


	%%WYKRES Czasy działania algorytmów dla $d=0.6$
	\begin{figure}[H]
		\centering
		\subsection*{Wykres przedstawiający czasy działania algorytmów dla $d=0.6$ }
		\begin{tikzpicture}
		\begin{axis}[
		width=0.9\textwidth,
		height = 0.5\textwidth,
		xlabel={Liczba wierzchołków},
		ylabel={Czas działania w misekundach},
		%
		scaled x ticks = false,
		xtick distance = 5,
		x tick label style={/pgf/number format/fixed},
		xticklabel style = {rotate= 90},
		x label style={at={(axis description cs:0.5,-0.15)},anchor=north},
		%%
		ytick distance = 500,
		scaled y ticks = false,
		y tick label style={/pgf/number format/fixed},
		y label style={at={(axis description cs:-0.05,0.85)},anchor=east},
		%%
		legend pos=north east,
		ymajorgrids=true,
		grid style=dashed,
		]
		%%
		\addplot[
		color=black,
		mark=*,
		]
		coordinates {
(5,0.001706)(6,0.001706)(7,0.001707)(8,0.002276)(9,0.002275)(10,0.002844)(11,0.003982)(12,0.003982)(13,0.003983)(14,0.003982)(20,0.006827)(25,0.010808)(30,0.014791)(35,0.019911)(40,0.0256)(45,0.031289)(50,0.039823)(55,0.046649)(60,0.054613)(65,0.065422)(70,0.074525)(75,0.088177)(80,0.095573)(85,0.10752)(90,0.120604)(95,0.135965)(100,0.199111)

		};
		%%
		\addplot[
		color=blue,
		mark=triangle,
		]
		coordinates {
(5,0.002845)(6,0.004551)(7,0.04608)(8,0.114347)(9,0.012516)(10,0.01536)(11,0.289564)(12,0.065422)(13,0.01024)(14,0.016497)(20,0.036978)(25,0.391395)(30,0.0512)(35,0.044374)(40,0.077369)(45,0.130275)(50,0.15872)(55,1.07975)(60,0.190008)(65,0.231537)(70,0.386275)(75,1.54908)(80,0.788479)(85,0.92615)(90,0.550684)(95,0.663892)(100,3.55214)

		};
		%%
\addplot[
color=red,
mark=square,
]
coordinates {
(5,0.003414)(6,0.017066)(7,0.051768)(8,0.098987)(9,0.897706)(10,6.12579)(11,28.6811)(12,171.07)(13,1486.26)(14,8225.92)	
};
		\legend{
			$t_{E}=f(n)$,
			$t_{H1} = f(n)$,
			$t_{HA} = f(n)$,
		}
		%%
		\end{axis}
		\end{tikzpicture}
	\end{figure}


	%%WYKRES Czasy działania algorytmów dla $d=0.6$ logarytmiczna
\begin{figure}[H]
	\centering
	\subsection*{Wykres przedstawiający czasy działania algorytmów dla $d=0.6$ skala logarytmiczna }
	\begin{tikzpicture}
	\begin{axis}[
	width=0.9\textwidth,
	height = 0.5\textwidth,
	xlabel={Liczba wierzchołków},
	ylabel={Czas działania w misekundach},
	%
	scaled x ticks = false,
	xtick distance = 5,
	x tick label style={/pgf/number format/fixed},
	xticklabel style = {rotate= 90},
	x label style={at={(axis description cs:0.5,-0.15)},anchor=north},
	%%
	ymode = log,
	ytick distance = 10,
	scaled y ticks = false,
	y tick label style={/pgf/number format/fixed},
	y label style={at={(axis description cs:-0.05,0.85)},anchor=east},
	%%
	legend pos=north east,
	ymajorgrids=true,
	grid style=dashed,
	]
	%%
	\addplot[
	color=black,
	mark=*,
	]
	coordinates {
(5,0.001706)(6,0.001706)(7,0.001707)(8,0.002276)(9,0.002275)(10,0.002844)(11,0.003982)(12,0.003982)(13,0.003983)(14,0.003982)(20,0.006827)(25,0.010808)(30,0.014791)(35,0.019911)(40,0.0256)(45,0.031289)(50,0.039823)(55,0.046649)(60,0.054613)(65,0.065422)(70,0.074525)(75,0.088177)(80,0.095573)(85,0.10752)(90,0.120604)(95,0.135965)(100,0.199111)

	};
	%%
	\addplot[
	color=blue,
	mark=triangle,
	]
	coordinates {
(5,0.002845)(6,0.004551)(7,0.04608)(8,0.114347)(9,0.012516)(10,0.01536)(11,0.289564)(12,0.065422)(13,0.01024)(14,0.016497)(20,0.036978)(25,0.391395)(30,0.0512)(35,0.044374)(40,0.077369)(45,0.130275)(50,0.15872)(55,1.07975)(60,0.190008)(65,0.231537)(70,0.386275)(75,1.54908)(80,0.788479)(85,0.92615)(90,0.550684)(95,0.663892)(100,3.55214)

	};
	%%
	\addplot[
	color=red,
	mark=square,
	]
	coordinates {
(5,0.003414)(6,0.017066)(7,0.051768)(8,0.098987)(9,0.897706)(10,6.12579)(11,28.6811)(12,171.07)(13,1486.26)(14,8225.92)

	};
	\legend{
		$t_{E}=f(n)$,
		$t_{H1} = f(n)$,
		$t_{HA} = f(n)$,
	}
	%%
	\end{axis}
	\end{tikzpicture}
\end{figure}

	Problemy znajdowania cyklu Eulera i cyklu Hamiltona dotyczą przeszukiwania grafu.
	
	Znajdowanie cyklu Eulera należy do klasy problemów łatwych (P), czyli takich dla których potrafimy znaleźć algorytm rozwiązujący ten problem w czasie wielomianowym. 
	
	Znajdowanie cyklu Hamiltona należy do problemów NP-zupełnych, które są podklasą problemów trudnych (NP), dla problemów które należą do klasy NP nie znamy rozwiązań działających w czasie wielomianowym lub mniejszym, czyli są to zadania o o złożoności co najmniej wykładniczej. Do problemów NP-zupełnych transformują się wielomianowo wszystkie problem z klasy NP. Rozwiązując problem NP-zupełny rozwiązujemy wszystkie problemy z tej podklasy, dlatego znajdując rozwiązanie jednego takiego problemu w czasie wielomianowym, znajdziemy rozwiązanie wielomianowe dla wszystkich problemów NP-zupełnych.
	
	Złożoność obliczeniowa algorytmu znajdowania cyklu Eulera wynosi $O(m)$, gdzie m - liczba krawędzi, ponieważ podczas przeszukiwania grafu trzeba przejść po wszystkich krawędziach.
	
	Złożoność obliczeniowa algorytmu znajdowania pojedynczego cyklu Hamiltona wynosi $O(n!)$, gdzie n - liczba wierzchołków, ponieważ w najgorszym przypadku należy sprawdzić wszystkie możliwe permutacje, a dla wszystkich cykli $O(n \cdot n!)$ .
	

	
\section{Czasy poszukiwania cyklu Eulera dla różnych wartości $d$}


%%Tabela przedstawiająca $T_{E}$ dla różnych wartości $d$
\begin{spacing}{1.3}
	{
		\centering
		\subsection*{Tabela przedstawiająca $T_{E}$ dla różnych wartości $d$ }
	}
	\begin{equation*}
	\begin{array}{|r|r|r|}
	\hline
	\multicolumn{1}{|c|}{$Liczba$}&
	\multicolumn{1}{c|}{t_{E} $ dla$}&
	\multicolumn{1}{c|}{t_{E} $ dla$}\\
	\multicolumn{1}{|c|}{$wierzchołków$}&
	\multicolumn{1}{c|}{d = 0.2 \quad [ms]}&
	\multicolumn{1}{c|}{d= 0.6\quad [ms]}\\
	\hline
	5&0.001&0.001\\
	6&0.001&0.001\\
	7&0.001&0.001\\
	8&0.001&0.002\\
	9&0.001&0.002\\
	10&0.001&0.002\\
	11&0.002&0.003\\
	12&0.002&0.003\\
	13&0.002&0.003\\
	14&0.003&0.003\\
	20&0.005&0.006\\
	25&0.007&0.010\\
	30&0.013&0.014\\
	35&0.015&0.019\\
	40&0.016&0.025\\
	45&0.022&0.031\\
	50&0.026&0.039\\
	55&0.030&0.046\\
	60&0.035&0.054\\
	65&0.041&0.065\\
	70&0.048&0.074\\
	75&0.055&0.088\\
	80&0.062&0.095\\
	85&0.073&0.107\\
	90&0.077&0.120\\
	95&0.088&0.135\\
	100&0.096&0.199\\\hline
	\end{array}
	\end{equation*}
\end{spacing}


	%%WYKRES Tabela przedstawiająca $T_{E}$ dla różnych wartości $d$
\begin{figure}[H]
		\centering
		\subsection*{Wykres przedstawiający $T_{E}$ dla różnych wartości $d$ }
	\begin{tikzpicture}
	\begin{axis}[
	width=0.9\textwidth,
	height = 0.5\textwidth,
	xlabel={Liczba wierzchołków},
	ylabel={Czas działania w misekundach},
	%
	scaled x ticks = false,
	xtick distance = 5,
	x tick label style={/pgf/number format/fixed},
	xticklabel style = {rotate= 90},
	x label style={at={(axis description cs:0.5,-0.15)},anchor=north},
	%%
	ytick distance = 0.05,
	scaled y ticks = false,
	y tick label style={/pgf/number format/fixed},
	y label style={at={(axis description cs:-0.05,0.85)},anchor=east},
	%%
	legend pos=north west,
	ymajorgrids=true,
	grid style=dashed,
	]
	%%
	\addplot[
	color=violet,
	mark=diamond,
	]
	coordinates {
(5,0.001707)(6,0.001138)(7,0.001707)(8,0.001707)(9,0.001706)(10,0.001707)(11,0.002844)(12,0.002275)(13,0.002276)(14,0.003413)(20,0.00512)(25,0.007395)(30,0.013085)(35,0.015929)(40,0.016497)(45,0.022187)(50,0.026168)(55,0.030151)(60,0.035271)(65,0.041528)(70,0.048355)(75,0.055183)(80,0.062577)(85,0.073956)(90,0.077369)(95,0.088747)(100,0.096711)
	};
	%%
	\addplot[
	color=black,
	mark=*,
	]
	coordinates {
(5,0.001706)(6,0.001706)(7,0.001707)(8,0.002276)(9,0.002275)(10,0.002844)(11,0.003982)(12,0.003982)(13,0.003983)(14,0.003982)(20,0.006827)(25,0.010808)(30,0.014791)(35,0.019911)(40,0.0256)(45,0.031289)(50,0.039823)(55,0.046649)(60,0.054613)(65,0.065422)(70,0.074525)(75,0.088177)(80,0.095573)(85,0.10752)(90,0.120604)(95,0.135965)(100,0.199111)

		
	};
	\legend{
		$d =0.2$,
		$d= 0.6$,
	}
	%%
	\end{axis}
	\end{tikzpicture}
\end{figure}

	%%WYKRES Tabela przedstawiająca $T_{E}$ dla różnych wartości $d$ logarytmiczna
\begin{figure}[H]
	\centering
	\subsection*{Wykres przedstawiający $T_{E}$ dla różnych wartości $d$ skala logarytmiczna }
	\begin{tikzpicture}
	\begin{axis}[
	width=0.9\textwidth,
	height = 0.5\textwidth,
	xlabel={Liczba wierzchołków},
	ylabel={Czas działania w misekundach},
	%
	scaled x ticks = false,
	xtick distance = 5,
	x tick label style={/pgf/number format/fixed},
	xticklabel style = {rotate= 90},
	x label style={at={(axis description cs:0.5,-0.15)},anchor=north},
	%%
	ymode = log,
	ytick distance = 0.000000000000000000000000000001,
	scaled y ticks = false,
	y tick label style={/pgf/number format/fixed},
	y label style={at={(axis description cs:-0.05,0.85)},anchor=east},
	%%
	legend pos=north west,
	ymajorgrids=true,
	grid style=dashed,
	]
	%%
	\addplot[
	color=violet,
	mark=diamond,
	]
	coordinates {
		(5,0.001707)(6,0.001138)(7,0.001707)(8,0.001707)(9,0.001706)(10,0.001707)(11,0.002844)(12,0.002275)(13,0.002276)(14,0.003413)(20,0.00512)(25,0.007395)(30,0.013085)(35,0.015929)(40,0.016497)(45,0.022187)(50,0.026168)(55,0.030151)(60,0.035271)(65,0.041528)(70,0.048355)(75,0.055183)(80,0.062577)(85,0.073956)(90,0.077369)(95,0.088747)(100,0.096711)
	};
	%%
	\addplot[
	color=black,
	mark=*,
	]
	coordinates {
		(5,0.001706)(6,0.001706)(7,0.001707)(8,0.002276)(9,0.002275)(10,0.002844)(11,0.003982)(12,0.003982)(13,0.003983)(14,0.003982)(20,0.006827)(25,0.010808)(30,0.014791)(35,0.019911)(40,0.0256)(45,0.031289)(50,0.039823)(55,0.046649)(60,0.054613)(65,0.065422)(70,0.074525)(75,0.088177)(80,0.095573)(85,0.10752)(90,0.120604)(95,0.135965)(100,0.199111)
		
		
	};
	\legend{
		$d =0.2$,
		$d= 0.6$,
	}
	%%
	\end{axis}
	\end{tikzpicture}
\end{figure}


Metoda poszukiwania cyklu Eulera oparta jest na algorytmie DFS (przeszukiwanie w głąb), z tą różnicą że przegląda krawędzi zamiast wierzchołków. Do przedstawienia grafu użyto macierzy sąsiedztwa {\Huge \textcolor{red}{TU MI SIE TŁUMACZ CZEMU MACIERZ SĄSIEDZTWA I CZY REPREZENTACJA MA WPŁYW NA ZŁOŻONOŚĆ OBLICZONIOWĄ METODY}}.

{\Huge \textcolor{red}{TUTAJ POPROSZĘ OPIS DZIAŁANIA ALGORYTMU ZGODZNIE Z IMPLEMENTACJĄ (MOZESZ UDAWAĆ ŻE TAKA BYŁA IMPLEMENTAJA) PRZYPOMNĘ TYLKO ŻE CHODZI O POSZUKIWANIE CYKLU EULERA, A NIE JAKIEGOŚ LOSOWEGO JAK OSTATNIO}}

Warunek konieczny i dostateczny istnienia cyklu Eulera w grafie:
\begin{spacing}{0.5}
	\begin{itemize}
		\item graf jest spójny,
		\item dla grafu nieskierowanego, wszystkie wierzchołki są stopnia parzystego,
		\item dla grafu skierowanego, taka sama liczba krawędzi wchodzących i wychodzących dla każdego wierzchołka.
	\end{itemize}
\end{spacing}

W testowanych grafach istniał cykl Eulera ponieważ zostały one wygenerowane odpowiednią metodą.
Opierała się ona na tworzeniu klik o rozmiarze 3, po stworzeniu pierwszej wybierany był losowy należący do grafu wierzchołek, oraz losowano dwa nie należące do grafu wierzchołki, z tych trzech wierzchołków do grafu dołączana była nowa klika. Dołączanie nowych klik trwało aż do osiągnięcia pożądanej gęstości.

{\huge zachowanie}




\section{Czasy poszukiwania pojedynczego i wszystkich cykli Hamiltona dla różnych wartości $d$}


%%Tabela prezentująca $t_{H1}$ i $t_{HA}$ dla różnych wartości d
\begin{spacing}{1.5}
	{\centering \subsection*{Tabela prezentująca $t_{H1}$ i $t_{HA}$ dla różnych wartości d}}
	\begin{equation*}
	\begin{array}{|r|r|r|r|r|}
	\hline
	\multicolumn{1}{|c|}{$Liczba$}&
	\multicolumn{2}{c|}{d = 0.2}&
	\multicolumn{2}{c|}{d = 0.6}\\\cline{2-5}
	\multicolumn{1}{|c|}{$wierzchołków$}&
	\multicolumn{1}{c|}{t_{H1} \quad [ms]}&
	\multicolumn{1}{c|}{t_{HA} \quad [ms]}&
	\multicolumn{1}{c|}{t_{H1} \quad [ms]}&
	\multicolumn{1}{c|}{t_{HA} \quad [ms]}\\
	\hline
	\end{array}
	\end{equation*}
\end{spacing}


	%%Wykres przedstawiający $t_{H1}$ dla różnych wartości d
\begin{figure}[H]
	
	\centering \subsection*{Wykres przedstawiający $t_{H1}$ dla różnych wartości d}
	\begin{tikzpicture}
	\begin{axis}[
	width=0.9\textwidth,
	height = 0.5\textwidth,
	xlabel={Liczba wierzchołków},
	ylabel={Czas działania w misekundach},
	%
	scaled x ticks = false,
	xtick distance = 200,
	x tick label style={/pgf/number format/fixed},
	xticklabel style = {rotate= 90},
	x label style={at={(axis description cs:0.5,-0.15)},anchor=north},
	%%
	ytick distance = 100,
	scaled y ticks = false,
	y tick label style={/pgf/number format/fixed},
	y label style={at={(axis description cs:-0.05,0.85)},anchor=east},
	%%
	legend pos=north west,
	ymajorgrids=true,
	grid style=dashed,
	]
	%%
	\addplot[
	color=black,
	mark=*,
	]
	coordinates {
		(200,2.172)(400,7.830)(600,17.860)(800,31.477)(1000,55.936)(1200,70.922)(1400,91.161)(1600,117.204)(1800,175.443)(2000,183.787)(2200,222.953)
	};
	%%
	\addplot[
	color=blue,
	mark=triangle,
	]
	coordinates {
		(200,4.841)(400,15.622)(600,38.984)(800,72.013)(1000,92.017)(1200,135.757)(1400,179.157)(1600,265.713)(1800,295.831)(2000,367.042)(2200,616.094)
		
	};
	\legend{
		$d =0.2$,
		$d= 0.6$,
	}
	%%
	\end{axis}
	\end{tikzpicture}
\end{figure}

		%%WYKRES Tabela prezentująca  $t_{HA}$ dla różnych wartości d
	\begin{figure}[H]
		
		\centering \subsection*{Tabela prezentująca  $t_{HA}$ dla różnych wartości d}
		\begin{tikzpicture}
		\begin{axis}[
		width=0.9\textwidth,
		height = 0.5\textwidth,
		xlabel={Liczba wierzchołków},
		ylabel={Czas działania w misekundach},
		%
		scaled x ticks = false,
		xtick distance = 200,
		x tick label style={/pgf/number format/fixed},
		xticklabel style = {rotate= 90},
		x label style={at={(axis description cs:0.5,-0.15)},anchor=north},
		%%
		ytick distance = 100,
		scaled y ticks = false,
		y tick label style={/pgf/number format/fixed},
		y label style={at={(axis description cs:-0.05,0.85)},anchor=east},
		%%
		legend pos=north west,
		ymajorgrids=true,
		grid style=dashed,
		]
		%%
		\addplot[
		color=black,
		mark=*,
		]
		coordinates {
			(200,2.172)(400,7.830)(600,17.860)(800,31.477)(1000,55.936)(1200,70.922)(1400,91.161)(1600,117.204)(1800,175.443)(2000,183.787)(2200,222.953)
		};
		%%
		\addplot[
		color=blue,
		mark=triangle,
		]
		coordinates {
			(200,4.841)(400,15.622)(600,38.984)(800,72.013)(1000,92.017)(1200,135.757)(1400,179.157)(1600,265.713)(1800,295.831)(2000,367.042)(2200,616.094)
			
		};
		\legend{
			$d =0.2$,
			$d= 0.6$,
		}
		%%
		\end{axis}
		\end{tikzpicture}
	\end{figure}

%%Tabela prezentująca liczbę cykli Hamiltona dla różnych wartości d
\begin{spacing}{1.5}
	{\centering \subsection*{Tabela prezentująca liczbę cykli Hamiltona dla różnych wartości d}}
	\begin{equation*}
	\begin{array}{|r|r|r|}
	\hline
	\multicolumn{1}{|c|}{$Liczba$}&
	\multicolumn{1}{c|}{$Liczba cykli$}&
	\multicolumn{1}{c|}{$Liczba cykli$}\\
	\multicolumn{1}{|c|}{$wierzchołków$}&
	\multicolumn{1}{c|}{$Hamiltona dla $ d=0.2}&
	\multicolumn{1}{c|}{$Hamiltona dla $ d=0.6}\\
	\hline
	\end{array}
	\end{equation*}
\end{spacing}

\end{spacing}
	\newpage
	\tableofcontents
\end{document}


