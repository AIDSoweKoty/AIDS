\documentclass[polish,polish,a4paper]{article}
\usepackage[T1]{fontenc}
\usepackage[utf8]{inputenc}
\usepackage{pgfplots}
\usepackage{pslatex}
\usepackage{setspace}
\usepackage{caption}
\usepackage{amssymb}
\usepackage{amsmath}
\usepackage{anysize}
\usepackage{graphicx}
\usepackage{hyperref}
\usepackage{float}
\usepackage{color}
\usepackage{subcaption}
\usepackage[polish]{babel}
\hypersetup{
	colorlinks=true,
	linkcolor=blue,
	filecolor=blue,      
	urlcolor=blue,
}

\marginsize{2.5cm}{2.5cm}{2cm}{2cm}


\begin{document}
	
		\begin{titlepage}
			\begin{flushright}
				{ Wtorki 16:50\\
					Grupa I3\\
					Kierunek Informatyka\\
					Wydział Informatyki\\
					Politechnika Poznańska}
			\end{flushright}
		\vspace*{\fill}
		\begin{center}
			{\Large Algorytmy i struktury danych \\[0.1cm]
				Sprawozdanie z zadania w zespołach nr. 4\\[0.1cm]
				prowadząca: dr hab. inż. Małgorzata Sterna, prof PP}\\
			{\Huge Algorytmy z powracaniem\\ [0.4cm]}
			{\large autorzy:\\[0.1cm]}
			{\large Piotr Więtczak nr indeksu 132339\\[0.1cm] Tomasz Chudziak nr indeksu 136691}\\[0.5cm]
			\today
		\end{center}
		\vspace*{\fill}
	\end{titlepage}

\begin{spacing}{1.25}

\section{Opis implementacji }
 
 Do implementacji algorytmów poszukujących cyklu Eulera $(E)$, pojedynczego cyklu Hamiltona $(H1)$ i wszystkich cykli Hamiltona użyliśmy języka C++. Do pomiarów czasu wykorzystaliśmy klasę $std :: chrono :: high\_resolution\_clock$ z biblioteki $chrono$. Do reprezentacji grafu zastosowaliśmy macierz sąsiedztwa, ze względu na
 intuicyjne działanie oraz łatwość implementacj.

\section{Czasy działania algorytmów}


	%% TABELA CZASÓW DZIAŁANIA
\begin{spacing}{1.3}
	
	{
		\centering
		\subsection*{Tabela przedstawiająca czasy działania algorytmów }
	}
	
	\begin{equation*}
	\begin{array}{|r|r|r|r|}
	\hline
	\multicolumn{1}{|c|}{$Liczba$}&
	\multicolumn{1}{c|}{t_{E}$ dla$ }&
	\multicolumn{1}{c|}{t_{H1} $ dla$}&
	\multicolumn{1}{c|}{t_{HA}$ dla$}\\
	\multicolumn{1}{|c|}{$wierzchołków$}&
	\multicolumn{1}{c|}{d = 0.6 \quad [ms]}&
	\multicolumn{1}{c|}{d = 0.6 \quad [ms]}&
	\multicolumn{1}{c|}{d = 0.6 \quad [ms]}\\
	\hline
5&0.001&0.003&0.004\\
6&0.001&0.005&0.021\\
7&0.001&0.031&0.037\\
8&0.001&0.033&0.072\\
9&0.001&0.006&0.396\\
10&0.002&0.130&3.945\\
11&0.002&0.005&25.734\\
12&0.003&0.048&232.968\\
13&0.003&0.009&1063.510\\
14&0.005&0.015&5300.210\\
15&0.005&0.010&\multicolumn{1}{|c|}{przerwano}\\
20&0.007&0.035&\multicolumn{1}{|c|}{przerwano}\\
25&0.010&11.446&\multicolumn{1}{|c|}{przerwano}\\
30&0.014&0.039&\multicolumn{1}{|c|}{przerwano}\\
35&0.019&0.061&\multicolumn{1}{|c|}{przerwano}\\
40&0.026&0.079&\multicolumn{1}{|c|}{przerwano}\\
45&0.032&0.101&\multicolumn{1}{|c|}{przerwano}\\
50&0.039&0.361&\multicolumn{1}{|c|}{przerwano}\\
55&0.046&0.131&\multicolumn{1}{|c|}{przerwano}\\
60&0.055&8.360&\multicolumn{1}{|c|}{przerwano}\\
65&0.063&5.921&\multicolumn{1}{|c|}{przerwano}\\
70&0.073&0.480&\multicolumn{1}{|c|}{przerwano}\\
75&0.089&1.922&\multicolumn{1}{|c|}{przerwano}\\
80&0.096&0.939&\multicolumn{1}{|c|}{przerwano}\\
85&0.116&5.121&\multicolumn{1}{|c|}{przerwano}\\
90&0.120&0.731&\multicolumn{1}{|c|}{przerwano}\\
95&0.174&502.357&\multicolumn{1}{|c|}{przerwano}\\
100&0.148&0.795&\multicolumn{1}{|c|}{przerwano}\\
\hline
	\end{array}
	\end{equation*}
\end{spacing}


	%%WYKRES Czasy działania algorytmów dla $d=0.6$
	\begin{figure}[H]
		\centering
		\subsection*{Wykres przedstawiający czasy działania algorytmów dla $d=0.6$ }
		\begin{tikzpicture}
		\begin{axis}[
		width=0.9\textwidth,
		height = 0.5\textwidth,
		xlabel={Liczba wierzchołków},
		ylabel={Czas działania w misekundach},
		%
		scaled x ticks = false,
		xtick distance = 5,
		x tick label style={/pgf/number format/fixed},
		xticklabel style = {rotate= 90},
		x label style={at={(axis description cs:0.5,-0.15)},anchor=north},
		%%
		ytick distance = 500,
		scaled y ticks = false,
		y tick label style={/pgf/number format/fixed},
		y label style={at={(axis description cs:-0.05,0.85)},anchor=east},
		%%
		legend pos=north east,
		ymajorgrids=true,
		grid style=dashed,
		]
		%%
		\addplot[
		color=black,
		mark=*,
		]
		coordinates {
(5,0.001706)(6,0.001137)(7,0.001706)(8,0.001707)(9,0.001706)(10,0.002845)(11,0.002844)(12,0.003982)(13,0.003982)(14,0.00512)(15,0.00512)(20,0.007396)(25,0.010809)(30,0.014222)(35,0.019342)(40,0.026169)(45,0.032995)(50,0.039822)(55,0.046648)(60,0.055751)(65,0.063715)(70,0.073387)(75,0.089315)(80,0.096142)(85,0.116053)(90,0.120604)(95,0.174079)(100,0.14848)


		};
		%%
		\addplot[
		color=blue,
		mark=triangle,
		]
		coordinates {
(5,0.003982)(6,0.00512)(7,0.031289)(8,0.033564)(9,0.006258)(10,0.130844)(11,0.00512)(12,0.048355)(13,0.009671)(14,0.01536)(15,0.01024)(20,0.035271)(25,11.446)(30,0.039822)(35,0.06144)(40,0.079644)(45,0.101831)(50,0.361813)(55,0.131413)(60,8.36038)(65,5.92156)(70,0.480142)(75,1.92284)(80,0.939803)(85,5.12113)(90,0.73159)(95,502.357)(100,0.795874)


		};
		%%
\addplot[
color=red,
mark=square,
]
coordinates {
(5,0.004552)(6,0.021049)(7,0.037547)(8,0.072249)(9,0.396515)(10,3.94581)(11,25.7342)(12,232.968)(13,1063.51)(14,5300.21)%(15,-)(20,-)(25,-)(30,-)(35,-)(40,-)(45,-)(50,-)(55,-)(60,-)(65,-)(70,-)(75,-)(80,-)(85,-)(90,-)(95,-)(100,-)

};
		\legend{
			$t_{E}=f(n)$,
			$t_{H1} = f(n)$,
			$t_{HA} = f(n)$,
		}
		%%
		\end{axis}
		\end{tikzpicture}
	\end{figure}


	%%WYKRES Czasy działania algorytmów dla $d=0.6$ logarytmiczna
\begin{figure}[H]
	\centering
	\subsection*{Wykres przedstawiający czasy działania algorytmów dla $d=0.6$ skala logarytmiczna }
	\begin{tikzpicture}
	\begin{axis}[
	width=0.9\textwidth,
	height = 0.5\textwidth,
	xlabel={Liczba wierzchołków},
	ylabel={Czas działania w misekundach},
	%
	scaled x ticks = false,
	xtick distance = 5,
	x tick label style={/pgf/number format/fixed},
	xticklabel style = {rotate= 90},
	x label style={at={(axis description cs:0.5,-0.15)},anchor=north},
	%%
	ymode = log,
	ytick distance = 10,
	scaled y ticks = false,
	y tick label style={/pgf/number format/fixed},
	y label style={at={(axis description cs:-0.05,0.85)},anchor=east},
	%%
	legend pos=south east,
	ymajorgrids=true,
	grid style=dashed,
	]
	%%
	\addplot[
	color=black,
	mark=*,
	]
	coordinates {
(5,0.001706)(6,0.001137)(7,0.001706)(8,0.001707)(9,0.001706)(10,0.002845)(11,0.002844)(12,0.003982)(13,0.003982)(14,0.00512)(15,0.00512)(20,0.007396)(25,0.010809)(30,0.014222)(35,0.019342)(40,0.026169)(45,0.032995)(50,0.039822)(55,0.046648)(60,0.055751)(65,0.063715)(70,0.073387)(75,0.089315)(80,0.096142)(85,0.116053)(90,0.120604)(95,0.174079)(100,0.14848)


	};
	%%
	\addplot[
	color=blue,
	mark=triangle,
	]
	coordinates {
(5,0.003982)(6,0.00512)(7,0.031289)(8,0.033564)(9,0.006258)(10,0.130844)(11,0.00512)(12,0.048355)(13,0.009671)(14,0.01536)(15,0.01024)(20,0.035271)(25,11.446)(30,0.039822)(35,0.06144)(40,0.079644)(45,0.101831)(50,0.361813)(55,0.131413)(60,8.36038)(65,5.92156)(70,0.480142)(75,1.92284)(80,0.939803)(85,5.12113)(90,0.73159)(95,502.357)(100,0.795874)


	};
	%%
	\addplot[
	color=red,
	mark=square,
	]
	coordinates {
(5,0.004552)(6,0.021049)(7,0.037547)(8,0.072249)(9,0.396515)(10,3.94581)(11,25.7342)(12,232.968)(13,1063.51)(14,5300.21)%(15,-)(20,-)(25,-)(30,-)(35,-)(40,-)(45,-)(50,-)(55,-)(60,-)(65,-)(70,-)(75,-)(80,-)(85,-)(90,-)(95,-)(100,-)


	};
	\legend{
		$t_{E}=f(n)$,
		$t_{H1} = f(n)$,
		$t_{HA} = f(n)$,
	}
	%%
	\end{axis}
	\end{tikzpicture}
\end{figure}

	Problemy znajdowania cyklu Eulera i cyklu Hamiltona dotyczą przeszukiwania grafu.
	
	Znajdowanie cyklu Eulera należy do klasy problemów łatwych (P), czyli takich dla których potrafimy znaleźć algorytm rozwiązujący ten problem w czasie wielomianowym. 
	
	Złożoność obliczeniowa algorytmu znajdowania cyklu Eulera, przy zastosowaniu listy sąsiedztwa, wynosi $O(m)$, gdzie m - liczba krawędzi, ponieważ podczas przeszukiwania grafu trzeba przejść po wszystkich krawędziach, jest to złożoność wielomianowa.
	
	Znajdowanie cyklu Hamiltona należy do problemów NP-zupełnych, które są podklasą problemów trudnych (NP), dla problemów które należą do klasy NP i NP-zupełnych nie znamy rozwiązań działających w czasie wielomianowym lub mniejszym, czyli są to zadania o o złożoności co najmniej wykładniczej. Do problemów NP-zupełnych transformują się wielomianowo wszystkie problem z klasy NP. Rozwiązując problem NP-zupełny rozwiązujemy wszystkie problemy z tej podklasy, dlatego znajdując rozwiązanie jednego takiego problemu w czasie wielomianowym, znajdziemy rozwiązanie wielomianowe dla wszystkich problemów NP-zupełnych.
	

	
	Złożoność obliczeniowa algorytmu znajdowania pojedynczego i wszystkich cykli Hamiltona wynosi $O(n! \cdot n)$, gdzie n - liczba wierzchołków, ponieważ w najgorszym przypadku należy sprawdzić wszystkie możliwe permutacje, jest to złożoność wykładnicza.
	

	
\section{Czasy poszukiwania cyklu Eulera dla różnych wartości $d$}


%%Tabela przedstawiająca $T_{E}$ dla różnych wartości $d$
\begin{spacing}{1.3}
	{
		\centering
		\subsection*{Tabela przedstawiająca $T_{E}$ dla różnych wartości $d$ }
	}
	\begin{equation*}
	\begin{array}{|r|r|r|}
	\hline
	\multicolumn{1}{|c|}{$Liczba$}&
	\multicolumn{1}{c|}{t_{E} $ dla$}&
	\multicolumn{1}{c|}{t_{E} $ dla$}\\
	\multicolumn{1}{|c|}{$wierzchołków$}&
	\multicolumn{1}{c|}{d = 0.2 \quad [ms]}&
	\multicolumn{1}{c|}{d= 0.6\quad [ms]}\\
	\hline
5&0.001&0.001\\
6&0.001&0.001\\
7&0.001&0.001\\
8&0.001&0.001\\
9&0.001&0.001\\
10&0.002&0.002\\
11&0.001&0.002\\
12&0.002&0.003\\
13&0.002&0.003\\
14&0.002&0.005\\
15&0.002&0.005\\
20&0.004&0.007\\
25&0.007&0.010\\
30&0.009&0.014\\
35&0.017&0.019\\
40&0.016&0.026\\
45&0.020&0.032\\
50&0.025&0.039\\
55&0.030&0.046\\
60&0.036&0.055\\
65&0.041&0.063\\
70&0.047&0.073\\
75&0.054&0.089\\
80&0.060&0.096\\
85&0.076&0.116\\
90&0.080&0.120\\
95&0.086&0.174\\
100&0.096&0.148\\\hline
	\end{array}
	\end{equation*}
\end{spacing}


	%%WYKRES Tabela przedstawiająca $T_{E}$ dla różnych wartości $d$
\begin{figure}[H]
		\centering
		\subsection*{Wykres przedstawiający $T_{E}$ dla różnych wartości $d$ }
	\begin{tikzpicture}
	\begin{axis}[
	width=0.9\textwidth,
	height = 0.5\textwidth,
	xlabel={Liczba wierzchołków},
	ylabel={Czas działania w misekundach},
	%
	scaled x ticks = false,
	xtick distance = 5,
	x tick label style={/pgf/number format/fixed},
	xticklabel style = {rotate= 90},
	x label style={at={(axis description cs:0.5,-0.15)},anchor=north},
	%%
	ytick distance = 0.02,
	scaled y ticks = false,
	y tick label style={/pgf/number format/fixed},
	y label style={at={(axis description cs:-0.05,0.85)},anchor=east},
	%%
	legend pos=north west,
	ymajorgrids=true,
	grid style=dashed,
	]
	%%
	\addplot[
	color=violet,
	mark=diamond,
	]
	coordinates {
(5,0.001138)(6,0.001137)(7,0.001138)(8,0.001138)(9,0.001707)(10,0.002276)(11,0.001707)(12,0.002276)(13,0.002275)(14,0.002844)(15,0.002844)(20,0.004551)(25,0.007395)(30,0.009671)(35,0.017066)(40,0.016498)(45,0.02048)(50,0.025031)(55,0.030151)(60,0.036409)(65,0.041529)(70,0.047786)(75,0.054613)(80,0.060871)(85,0.076231)(90,0.080213)(95,0.086471)(100,0.09671)


	};
	%%
	\addplot[
	color=black,
	mark=*,
	]
	coordinates {
(5,0.001706)(6,0.001137)(7,0.001706)(8,0.001707)(9,0.001706)(10,0.002845)(11,0.002844)(12,0.003982)(13,0.003982)(14,0.00512)(15,0.00512)(20,0.007396)(25,0.010809)(30,0.014222)(35,0.019342)(40,0.026169)(45,0.032995)(50,0.039822)(55,0.046648)(60,0.055751)(65,0.063715)(70,0.073387)(75,0.089315)(80,0.096142)(85,0.116053)(90,0.120604)(95,0.174079)(100,0.14848)



		
	};
	\legend{
		$d =0.2$,
		$d= 0.6$,
	}
	%%
	\end{axis}
	\end{tikzpicture}
\end{figure}

Metoda poszukiwania cyklu Eulera oparta jest na algorytmie DFS (przeszukiwanie w głąb), z tą różnicą że przegląda krawędzie zamiast wierzchołków. Do przedstawienia grafu użyto macierzy sąsiedztwa. Uznaliśmy, że sprawi ona nam najmniej problemów związanych z implementacją. Niestety wybrana przez nas struktura nie jest najwydajniejsza do tego typu zadania, szybsza byłaby lista następników. Macierz sąsiedztwa posiada jednak przewagę w  krótkim czasie i łatwości usuwania krawędzi.  

Metoda poszukująca cyklu Eulera przechodzi przez każdą krawędź co najmniej raz. Dlatego dla listy następników złożoność wynosi $ O(m) $. Do implementacji z macierzą sąsiedztwa trzeba dodać czas wyszukiwania następnika dla wszystkich wierzchołków, stąd złożoność obliczeniowa dla tej implementacji wynosi $ O(n^{2} + m) $.

Algorytm rozpoczyna działanie od sprawdzenia, czy wszystkie wierzchołki są parzystego stopnia, w wypadku gdy nie są, kończy swoje działanie i zwraca odpowiednią wartość (w tym wypadku -1). Na tym etapie nie jest sprawdzana spójność grafu, ten warunek jest uwzględniany podczas jego tworzenia. Jeżeli struktura przejdzie ten test, wybierany jest wierzchołek startowy (w tej implementacji wierzchołek o najniższym możliwym indeksie). Jest on kładziony na stos. Następnie wybierany jest wierzchołek o jak najmniejszym numerze, do którego istnieje połączenie z obecnej lokalizacji, przechodzi do niego, dodaje go na stos i usuwa połączenie między nimi. Algorytm powtarza to tak długo, aż nie znajdzie się w wierzchołku, z którego nie ma wyjścia. W ten czas zaczyna zdejmować ze stosu, wykonuje tę czynność tak długo, aż nie wróci do wierzchołka, w którym istnieje niewykorzystane połączenie. Jeżeli je znajdzie schemat postępowania powtarza się. Algorytm kończy się, gdy nie zostanie już żadna krawędź. Kolejność zdejmowania ze stosu jest tu bardzo istotna, to właśnie ona tworzy cykl Eulera.

Warunek konieczny i dostateczny istnienia cyklu Eulera w grafie:
\begin{spacing}{0.5}
	\begin{itemize}
		\item graf jest spójny,
		\item dla grafu nieskierowanego, wszystkie wierzchołki są stopnia parzystego,
		\item dla grafu skierowanego, taka sama liczba krawędzi wchodzących i wychodzących dla każdego wierzchołka.
	\end{itemize}
\end{spacing}

W testowanych grafach istniał cykl Eulera ponieważ zostały one wygenerowane odpowiednią metodą.
Opierała się ona na tworzeniu klik o rozmiarze 3, po stworzeniu pierwszej wybierany był losowy należący do grafu wierzchołek, oraz losowano dwa nienależące do grafu, z tych trzech wierzchołków do grafu dołączana była nowa klika. Dołączanie nowych klik trwało aż do osiągnięcia pożądanej gęstości.

Wraz z zwiększaniem liczby wierzchołków rośnie czas działania algorytmu poszukującego cyklu Eulera, tak samo dla grafów o tych samych rozmiarach ale o większej gęstości czas jest dłuższy, co oznacza że algorytm zachowuje się naturalnie.




\section{Czasy poszukiwania pojedynczego i wszystkich cykli Hamiltona dla różnych wartości $d$}


%%Tabela prezentująca $t_{H1}$ i $t_{HA}$ dla różnych wartości d
\begin{spacing}{1.3}
	{\centering \subsection*{Tabela prezentująca $t_{H1}$ i $t_{HA}$ dla różnych wartości d}}
	\begin{equation*}
	\begin{array}{|r|r|r|r|r|}
	\hline
	\multicolumn{1}{|c|}{$Liczba$}&
	\multicolumn{2}{c|}{d = 0.2}&
	\multicolumn{2}{c|}{d = 0.6}\\\cline{2-5}
	\multicolumn{1}{|c|}{$wierzchołków$}&
	\multicolumn{1}{c|}{t_{H1} \quad [ms]}&
	\multicolumn{1}{c|}{t_{HA} \quad [ms]}&
	\multicolumn{1}{c|}{t_{H1} \quad [ms]}&
	\multicolumn{1}{c|}{t_{HA} \quad [ms]}\\
	\hline
5&0.001&0.002&0.003&0.004\\
6&0.002&0.002&0.005&0.021\\
7&0.001&0.001&0.031&0.037\\
8&0.002&0.003&0.033&0.072\\
9&0.001&0.001&0.006&0.396\\
10&0.008&0.010&0.130&3.945\\
11&0.000&0.001&0.005&25.734\\
12&0.009&0.010&0.048&232.968\\
13&0.012&0.013&0.009&1063.510\\
14&0.039&0.043&0.015&5300.210\\
15&0.062&0.069&0.010&\multicolumn{1}{|c|}{przerwano}\\
20&18.071&19.434&0.035&\multicolumn{1}{|c|}{przerwano}\\
25&1.196&12220.800&11.446&\multicolumn{1}{|c|}{przerwano}\\
%30&0.000&\multicolumn{1}{|c|}{przerwano}&0.039&\multicolumn{1}{|c|}{przerwano}\\
%35&0.000&\multicolumn{1}{|c|}{przerwano}&0.061&\multicolumn{1}{|c|}{przerwano}\\
%40&0.000&\multicolumn{1}{|c|}{przerwano}&0.079&\multicolumn{1}{|c|}{przerwano}\\
%45&0.000&\multicolumn{1}{|c|}{przerwano}&0.101&\multicolumn{1}{|c|}{przerwano}\\
%50&0.000&\multicolumn{1}{|c|}{przerwano}&0.361&\multicolumn{1}{|c|}{przerwano}\\
%55&0.000&\multicolumn{1}{|c|}{przerwano}&0.131&\multicolumn{1}{|c|}{przerwano}\\
%60&0.000&\multicolumn{1}{|c|}{przerwano}&8.360&\multicolumn{1}{|c|}{przerwano}\\
%65&0.000&\multicolumn{1}{|c|}{przerwano}&5.921&\multicolumn{1}{|c|}{przerwano}\\
%70&0.000&\multicolumn{1}{|c|}{przerwano}&0.480&\multicolumn{1}{|c|}{przerwano}\\
%75&0.000&\multicolumn{1}{|c|}{przerwano}&1.922&\multicolumn{1}{|c|}{przerwano}\\
%80&0.000&\multicolumn{1}{|c|}{przerwano}&0.939&\multicolumn{1}{|c|}{przerwano}\\
%85&0.000&\multicolumn{1}{|c|}{przerwano}&5.121&\multicolumn{1}{|c|}{przerwano}\\
%90&0.000&\multicolumn{1}{|c|}{przerwano}&0.731&\multicolumn{1}{|c|}{przerwano}\\
%95&0.000&\multicolumn{1}{|c|}{przerwano}&502.357&\multicolumn{1}{|c|}{przerwano}\\
%100&0.000&\multicolumn{1}{|c|}{przerwano}&0.795&\multicolumn{1}{|c|}{przerwano}\\
\hline
	\end{array}
	\end{equation*}
\end{spacing}


	%%Wykres przedstawiający $t_{H1}$ dla różnych wartości d
\begin{figure}[H]
	
	\centering \subsection*{Wykres przedstawiający $t_{H1}$ dla różnych wartości d}
	\begin{tikzpicture}
	\begin{axis}[
	width=0.9\textwidth,
	height = 0.5\textwidth,
	xlabel={Liczba wierzchołków},
	ylabel={Czas działania w misekundach},
	%
	scaled x ticks = false,
	xtick distance = 5,
	x tick label style={/pgf/number format/fixed},
	xticklabel style = {rotate= 90},
	x label style={at={(axis description cs:0.5,-0.15)},anchor=north},
	%%
	ytick distance = 100,
	scaled y ticks = false,
	y tick label style={/pgf/number format/fixed},
	y label style={at={(axis description cs:-0.05,0.85)},anchor=east},
	%%
	legend pos=north west,
	ymajorgrids=true,
	grid style=dashed,
	]
	%%
	\addplot[
	color=black,
	mark=*,
	]
	coordinates {
(5,0.001707)(6,0.002275)(7,0.001138)(8,0.002276)(9,0.001138)(10,0.008533)(11,0.000569)(12,0.009102)(13,0.012515)(14,0.039254)(15,0.062577)(20,18.0713)(25,1.19694)%(30,0.000)(35,0.000)(40,0.000)(45,0.000)(50,0.000)(55,0.000)(60,0.000)(65,0.000)(70,0.000)(75,0.000)(80,0.000)(85,0.000)(90,0.000)(95,0.000)(100,0.000)
	};
	%%
	\addplot[
	color=blue,
	mark=triangle,
	]
	coordinates {
(5,0.003982)(6,0.00512)(7,0.031289)(8,0.033564)(9,0.006258)(10,0.130844)(11,0.00512)(12,0.048355)(13,0.009671)(14,0.01536)(15,0.01024)(20,0.035271)(25,11.446)%(30,0.039822)(35,0.06144)(40,0.079644)(45,0.101831)(50,0.361813)(55,0.131413)(60,8.36038)(65,5.92156)(70,0.480142)(75,1.92284)(80,0.939803)(85,5.12113)(90,0.73159)(95,502.357)(100,0.795874)

		
	};
	\legend{
		$d =0.2$,
		$d= 0.6$,
	}
	%%
	\end{axis}
	\end{tikzpicture}
\end{figure}

		%%WYKRES Tabela prezentująca  $t_{HA}$ dla różnych wartości d
	\begin{figure}[H]
		
		\centering \subsection*{Tabela prezentująca  $t_{HA}$ dla różnych wartości d}
		\begin{tikzpicture}
		\begin{axis}[
		width=0.9\textwidth,
		height = 0.5\textwidth,
		xlabel={Liczba wierzchołków},
		ylabel={Czas działania w misekundach},
		%
		scaled x ticks = false,
		xtick distance = 5,
		x tick label style={/pgf/number format/fixed},
		xticklabel style = {rotate= 90},
		x label style={at={(axis description cs:0.5,-0.15)},anchor=north},
		%%
		ytick distance = 1000,
		scaled y ticks = false,
		y tick label style={/pgf/number format/fixed},
		y label style={at={(axis description cs:-0.05,0.85)},anchor=east},
		%%
		legend pos=north west,
		ymajorgrids=true,
		grid style=dashed,
		]
		%%
		\addplot[
		color=black,
		mark=*,
		]
		coordinates {
(5,0.002275)(6,0.002276)(7,0.001137)(8,0.003414)(9,0.001707)(10,0.01024)(11,0.001137)(12,0.01024)(13,0.013654)(14,0.043236)(15,0.069405)(20,19.4349)(25,12220.8)%(30,-)(35,-)(40,-)(45,-)(50,-)(55,-)(60,-)(65,-)(70,-)(75,-)(80,-)(85,-)(90,-)(95,-)(100,-)

		};
		%%
		\addplot[
		color=blue,
		mark=triangle,
		]
		coordinates {
(5,0.004552)(6,0.021049)(7,0.037547)(8,0.072249)(9,0.396515)(10,3.94581)(11,25.7342)(12,232.968)(13,1063.51)(14,5300.21)%(15,-)(20,-)(25,-)(30,-)(35,-)(40,-)(45,-)(50,-)(55,-)(60,-)(65,-)(70,-)(75,-)(80,-)(85,-)(90,-)(95,-)(100,-)

			
		};
		\legend{
			$d =0.2$,
			$d= 0.6$,
		}
		%%
		\end{axis}
		\end{tikzpicture}
	\end{figure}

%%Tabela prezentująca liczbę cykli Hamiltona dla różnych wartości d
\begin{spacing}{1.3}
	{\centering \subsection*{Tabela prezentująca liczbę cykli Hamiltona dla różnych wartości d}}
	\begin{equation*}
	\begin{array}{|r|r|r|}
	\hline
	\multicolumn{1}{|c|}{$Liczba$}&
	\multicolumn{1}{c|}{$Liczba cykli$}&
	\multicolumn{1}{c|}{$Liczba cykli$}\\
	\multicolumn{1}{|c|}{$wierzchołków$}&
	\multicolumn{1}{c|}{$Hamiltona dla $ d=0.2}&
	\multicolumn{1}{c|}{$Hamiltona dla $ d=0.6}\\
	\hline
	5&0&0\\
	6&0&2\\
	7&0&0\\
	8&0&2\\
	9&0&4\\
	10&0&128\\
	11&0&3372\\
	12&0&11964\\
	13&0&66680\\
	14&0&346018\\
	15&0&\multicolumn{1}{|c|}{przerwano}\\
	20&0&\multicolumn{1}{|c|}{przerwano}\\
	25&0&\multicolumn{1}{|c|}{przerwano}\\
	
	\hline
	\end{array}
	\end{equation*}
\end{spacing}
Złożoność obliczeniowa obu algorytmów to $ O(n! \cdot n) $.
Jednakże istnieje duża różnica pomiędzy szukaniem jednego, a szukaniem wszystkich cykli w grafie. Ma ona bezpośrednie przełożenie na czas trwania programu. W drugim przypadku algorytm działa dłużej, ponieważ nie poprzestaje na tylko jednej odpowiedzi. W szczególnym przypadku czas trwania dla obu będzie taki sam, zdarza się to wtedy gdy w grafie nie ma cyklu. Czas trwania dla szukania pojedynczego cyklu uzależniony jest od danych wejściowych, natomiast dla wszystkich zawsze sprawdzane są wszystkie możliwe permutacje.
Do przeprowadzenia eksperymentu wykorzystaliśmy reprezentacje grafu w postaci macierzy sąsiedztwa. Została ona użyta z tych samych powodów co w przypadku algorytmu poszukującego cyklu Eulera. Jeżeli wykorzystalibyśmy listę łuków, program nie wykonywałby dodatkowych operacji, takich jak sprawdzanie, czy łuk istnieje.

Do znajdowania cykli Hamiltona w grafie używa się algorytmu z powracaniem. Algorytm poszukuje cyklu generując wszystkie przypadki, w tym przypadku są to tablice jednowymiarowe, które zapisuje do osobnej struktury odrzucając te które nie spełniają kryteriów. Ta metoda jest używana nie tylko do poszukiwaniu cykli Hamiltona, ale może być stosowana do rozwiązywania wszystkich problemów NP-zupełnych.

Zarówno liczba wierzchołków jak i krawędzi ma duży wpływ na działanie tej metody.
Dla grafów o większej liczbie wierzchołków algorytm będzie musiał przeanalizować większą liczbę możliwych przypadków, co wydłuży jego czas działania, szczególnie przy wyszukiwaniu wszystkich cykli. W grafach gęstszych poszukiwanie pojedynczego cyklu Hamiltona zajmie mniej czasu, ponieważ wzrosną stopnie wierzchołków i liczba cykli Hamiltona. Mniejsza gęstość grafu pozwoli na szybsze odrzucenie złych rozwiązań.
\end{spacing}
	\newpage
	\tableofcontents
\end{document}


